Meliton von Sardes († vor 190), Aus einer Osterpredigt

\vspace{10pt}

%\par \hfill(Nn. 65071: SC 123, 95-101)

\lettrine[lines=3]{D}{}ie Propheten haben vieles vorausverkündigt über das Paschamysterium, das Christus ist, „dem Ehre sei in alle Ewigkeit. Amen“. Er kam vom Himmel auf die Erde wegen des leidenden Menschen; den leidenden Menschen zog er wie ein Kleid an im Schoß der Jungfrau und ging hervor als Mensch; durch einen Leib, der dem Leiden ausgesetzt war, nahm er die Leiden des leidenden Menschen auf sich und vernichtete die Leiden des Fleisches. Durch den Geist aber, der nicht sterben konnte, tötete er den Mörder Tod.

Er wurde zum Schlachten geführt wie ein Lamm und getötet wie ein Schaf. Wie aus einem Ägypten erlöste er uns aus dem Dienst der Welt. Er rettete uns aus der Knechtschaft des Teufels wie aus der Hand des Pharaos; er besiegelte unsere Seelen mit seinem eigenen Geist und die Glieder unseres Leibes mit seinem Blut. Er ist es, der Verwirrung über den Tod brachte und den Teufel in Trauer versetzte wie Mose den Pharao. Er schlug die Bosheit und verdammte die Ungerechtigkeit zur Unfruchtbarkeit wie Mose Ägypten.

Er ist es, der uns der Knechtschaft entrissen und uns befreit hat, der uns aus der Finsternis zum Licht führte, vom Tod zum Leben, von der Gewaltherrschaft zu ewigem Königtum, der uns zu einer neuen Priesterschaft machte, zu einem erwählten und ewigen Volk. Er ist das Paschalamm unseres Heils. Er ertrug in vielen vieles: Er wurde in Abel gemordet. 

In Isaak wurden ihm die Füße gefesselt, in Jakob mußte er auswandern. In Josef wurde er verkauft, in Mose ausgesetzt, im Lamm geschlachtet, in David verfolgt, in den Propheten geschmäht. Er wurde Mensch in der Jungfrau, ans Holz gehängt, in das Grab der Erde gesenkt. Er erstand von den Toten und stieg empor zur Höhe des Himmels. Er ist das Lamm, das verstummt, aus der Herde geholt, zum Schlachten geführt, am Abend geopfert und in der Nacht begraben. Am Holz zerbrach man ihn nicht und im Grab verweste er nicht. Er stand von den Toten auf und erweckte den Menschen aus dem Grab der Unterwelt.
