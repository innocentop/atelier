\greannotation{I g}

\gabcsnippet{(c4)Dó(f)mi(gh)ni(h) est(h) ter(h)ra,(h) et(h) ple(h)ni(h)t<ul>ú</ul>(ixi)do(h) <ul>e</ul>(g)ius:(h) <v>$\star$</v>(:) or(h)bis(h) ter(h)rá(h)rum,(h) et(h) u(h)ni(h)vér(h)si(h) qui(h) há(h)bi(h)t<v>\uuline{a}</v>nt(g) in(f) <ul>e</ul>(gh)o.(g) (::)}

\vspace{5pt}

2. Quia ipse super mária fund\uline{á}vit \uline{e}um:~$\star$ et super flúmina præpar\uuline{á}vit \uline{e}um.

3. Quis ascéndet in m\uline{o}ntem D\uline{ó}mini?~$\star$ aut quis stabit in loco s\uuline{a}ncto \uline{e}ius?

4. Innocens mánibus et mundo cord\uuline{e},~† qui non accépit in vano \uline{á}nimam s\uline{u}am,~$\star$ nec iurávit in dolo próx\uuline{i}mo s\uline{u}o.

5. Hic accípiet benedicti\uline{ó}nem a D\uline{ó}mino:~$\star$ et misericórdiam a Deo, salut\uuline{á}ri s\uline{u}o.

6. Hæc est generátio quær\uline{é}ntium \uline{e}um,~$\star$ quæréntium fáciem D\uuline{e}i I\uline{a}cob.

7. Attóllite portas, príncipes, vestr\uuline{a}s,~† et elevámini, portæ \uline{æ}tern\uline{á}les:~$\star$ et introíb\uuline{i}t Rex gl\uline{ó}riæ.

8. Quis est iste Rex glór\uuline{i}æ?~† Dóminus f\uline{o}rtis et p\uline{o}tens:~$\star$ Dóminus pot\uuline{e}ns in pr\uline{ǽ}lio.

9. Attóllite portas, príncipes, vestr\uuline{a}s,~† et elevámini, portæ \uline{æ}tern\uline{á}les:~$\star$ et introíb\uuline{i}t Rex gl\uline{ó}riæ.

10. Quis est \uline{i}ste Rex gl\uline{ó}riæ?~$\star$ Dóminus virtútum ipse \uuline{e}st Rex gl\uline{ó}riæ.