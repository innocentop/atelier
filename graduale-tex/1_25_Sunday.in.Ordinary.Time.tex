\documentclass[11pt,twoside]{book}

%%Page Size (rev. 08/19/2016)
%\usepackage[inner=0.5in, outer=0.5in, top=0.5in, bottom=0.5in, papersize={6in,9in}, head=12pt, headheight=30pt, headsep=5pt]{geometry}
\usepackage[inner=0.5in, outer=0.5in, top=0.5in, bottom=0.3in, papersize={5.5in,8.5in}, head=12pt, headheight=30pt, headsep=5pt]{geometry}
%% width of textblock = 324 pt / 4.5in
%% A5 = 5.8 x 8.3 inches -- if papersize is A5, then margins should be [inner=0.75in, outer=0.55in, top=0.4in, bottom=0.4in]


%%Header (rev. 4/11/2011)
\usepackage{fancyhdr}
 \pagestyle{fancy}
\renewcommand{\chaptermark}[1]{\markboth{#1}{}}
\renewcommand{\sectionmark}[1]{\markright{\thesection\ #1}}
 \fancyhf{}
\fancyhead[LE,RO]{\thepage}
\fancyhead[CE]{Graduale O.P.}
\fancyhead[CO]{\leftmark}
 \fancypagestyle{plain}{ %
\fancyhf{} % remove everything
\renewcommand{\headrulewidth}{0pt} % remove lines as well
\renewcommand{\footrulewidth}{0pt}}



\usepackage[autocompile,allowdeprecated=false]{gregoriotex}
\usepackage{gregoriosyms}
\gresetgregoriofont[op]{greciliae}




%%Titles (rev. 9/4/2011) -- TOCLESS --- lets you have sections that don't appear in the table of contents

\setcounter{secnumdepth}{-2}

\usepackage[compact,nobottomtitles*]{titlesec}
\titlespacing*{\part}{0pt}{*0}{*1}
\titlespacing*{\chapter}{0pt}{*0}{*1}
\titlespacing*{\section}{0pt}{*0}{*1}
\titlespacing*{\subsubsection}{0pt}{*0}{*0}
\titlespacing*{\subsubsubsection}{0pt}{10pt}{*0}
\titleformat{\part} {\normalfont\Huge\sc\center}{\thepart}{1em}{}
\titleformat{\chapter} {\normalfont\huge\sc\center}{\thechapter}{1em}{}
\titleformat{\section} {\normalfont\LARGE\sc\center}{\thesection}{1em}{}
\titleformat{\subsection} {\normalfont\Large\sc\center}{\thesubsubsection}{1em}{}
\titleformat{\subsubsection}{\normalfont\large\sc\center}{\thesubsubsubsection}{1em}{}

\newcommand{\nocontentsline}[3]{}
\newcommand{\tocless}[2]{\bgroup\let\addcontentsline=\nocontentsline#1{#2}\egroup} %% lets you have sections that don't appear in the table of contents


%%%

%%Index (rev. December 11, 2013)
\usepackage[noautomatic,nonewpage]{imakeidx}


\makeindex[name=incipit,title=Index]
\indexsetup{level=\section,toclevel=section,noclearpage}

\usepackage[indentunit=8pt,rule=.5pt,columns=2]{idxlayout}


%%Table of Contents (rev. May 16, 2011)

%\usepackage{multicol}
%\usepackage{ifthen}
%\usepackage[toc]{multitoc}

%% General settings (rev. January 19, 2015)

\usepackage{ulem}

\usepackage[latin,english]{babel}
\usepackage{lettrine}


\usepackage{fontspec}

\setmainfont[Ligatures=TeX,BoldFont=MinionPro-Bold,ItalicFont=MinionPro-It, BoldItalicFont=MinionPro-BoldIt]{MinionPro-Regular-Modified.otf}


\usepackage{unicode-math}
\setmathfont{LatinModernMath-Regular}
\AtBeginDocument{\grelatexsimpledefbarredsymbol{V}{0.1em}{0.12em}{0.14em}{0.18em}}


%% Style for translation line
\grechangestyle{translation}{\textnormal\selectfont}
\grechangestyle{annotation}{\fontsize{10}{10}\selectfont}
\grechangestyle{commentary}{\textnormal\selectfont}
\gresetcustosalteration{invisible}



%\grechangedim{annotationseparation}{0.1cm}{scalable}

%\GreLoadSpaceConf{smith-four}

\frenchspacing

\usepackage{indentfirst} %%%indents first line after a section

\usepackage{graphicx}
%\usepackage{tocloft}

%%Hyperref (rev. August 20, 2011)
%\usepackage[colorlinks=false,hyperindex=true,bookmarks=true]{hyperref}
\usepackage{hyperref}
\hypersetup{pdftitle={Graduale O.P. 2016}}
\hypersetup{pdfauthor={Order of Preachers}}
\hypersetup{pdfsubject={Liturgy}}
\hypersetup{pdfkeywords={Dominican, Liturgy, Order of Preachers, Dominican Rite, Liturgia Horarum, Divine Office}}

\newlength{\drop}



\begin{document}


%%%Initial Matter within Body (20 May 2011)
\raggedbottom
\grechangedim{maxbaroffsettextleft}{0 cm}{scalable}


\chapter[25\textsuperscript{th} Sunday in Ordinary Time]{Graduale O.P.}
\section{25\textsuperscript{th} Sunday in Ordinary Time}

\begin{figure}[b]\centering\includegraphics[width=1.1in]{alteliersaintjacques-shell}\par \textsc{\Large{Atelier Saint-Jacques}}\par \vspace{3pt} (Revised: \today)  \vspace{5pt} \end{figure}

\newpage


%%Combination

\subsubsection{Officium}  \greannotation{IV} \index[Officium]{Salus populi} \label{Salus populi (Officium)} \grecommentary[0pt]{Cf. Ps 36:39, 40, 28; \Vbar. Ps 77:1} \gresetinitiallines{1} \grechangedim{maxbaroffsettextleft@nobar}{12 cm}{scalable} \grechangedim{spaceabovelines}{0.5cm}{scalable} \gresetlyriccentering{vowel}  \gregorioscore{graduale-chants/in--salus_populi--dominican--id_5388}  \vspace{5pt} \par{I am the salvation of the people, says the Lord; in whatever tribulation they shall cry to Me, I will hear them, and I will be their Lord forever. \Vbar. Attend, O My people, to My law; incline your ear to the words of My mouth.} \newpage
\subsubsection{Responsorium} Dom. anno A: \greannotation{V} \index[Responsorium]{Prope est Dominus} \label{Prope est Dominus (Responsorium)} \grecommentary[0pt]{Ps 144:18; \Vbar. 21} \gresetinitiallines{1} \grechangedim{maxbaroffsettextleft@nobar}{12 cm}{scalable} \grechangedim{spaceabovelines}{0.5cm}{scalable} \gresetlyriccentering{vowel}  \gregorioscore{graduale-chants/gr--prope_est_dominus--dominican--id_6717}  \vspace{5pt} \par{The Lord is near to all who call upon Him; to all who call upon Him in truth. \Vbar. My mouth shall speak the praise of the Lord; and let all flesh bless His holy name.} \newpage
\subsubsection{Responsorium} Dom. anno B et in feriis: \greannotation{VII} \index[Responsorium]{Dirigatur oratio mea} \label{Dirigatur oratio mea (Responsorium)} \grecommentary[3pt]{Ps 140:2} \gresetinitiallines{1} \grechangedim{maxbaroffsettextleft@nobar}{12 cm}{scalable} \grechangedim{spaceabovelines}{0.5cm}{scalable} \gresetlyriccentering{vowel}  \gregorioscore{graduale-chants/gr--dirigatur_oratio_mea--dominican--id_5920}  \vspace{5pt} \par{Let my prayer ascend like incense in Your presence, O Lord. \Vbar. May the lifting up of my hands be an evening sacrifice.} \newpage
\subsubsection{Responsorium} Dom. anno C: \greannotation{V} \index[Responsorium]{Quis sicut Dominus} \label{Quis sicut Dominus (Responsorium)} \grecommentary[0pt]{Ps 112:5, 6; \Vbar. 7} \gresetinitiallines{1} \grechangedim{maxbaroffsettextleft@nobar}{0 cm}{scalable} \grechangedim{spaceabovelines}{0.5cm}{scalable} \gresetlyriccentering{vowel}  \gregorioscore{graduale-chants/gr--quis_sicut_dominus--dominican--id_5616}  \vspace{5pt} \par{Who is like the Lord our God, who dwells on high, looking down on the lowly in heaven and on earth? \Vbar. Raising up the needy from the earth, and lifting up the poor out of the dunghill.} \newpage


\subsubsection{Alleluia}  \greannotation{II} \index[Alleluia]{Confitemini Domino et invocate} \label{Confitemini Domino et invocate (Alleluia)} \grecommentary[0pt]{Ps 104:1} \gresetinitiallines{1} \grechangedim{maxbaroffsettextleft@nobar}{12 cm}{scalable} \grechangedim{spaceabovelines}{0.5cm}{scalable} \gresetlyriccentering{vowel}  \gregorioscore{graduale-chants/al--confitemini_et_invocate--dominican--id_5157}  \vspace{5pt} \par{Give glory to the Lord and call upon His name; declare His deeds among the nations.} \newpage


\subsubsection{Offertorium}  \greannotation{VIII} \index[Offertorium]{Si ambulavero} \label{Si ambulavero (Offertorium)} \grecommentary[0pt]{Ps 137:7} \gresetinitiallines{1} \grechangedim{maxbaroffsettextleft@nobar}{12 cm}{scalable} \grechangedim{spaceabovelines}{0.5cm}{scalable} \gresetlyriccentering{vowel}  \gregorioscore{graduale-chants/of--si_ambulavero--dominican--id_5929}  \vspace{5pt} \par{If I shall walk in the midst of tribulation, You will revive me, O Lord; and You will stretch forth Your hand against the wrath of my enemies, just as Your right hand saved me.} \newpage
\subsubsection{Communio}  \greannotation{VII} \index[Communio]{Tu mandasti} \label{Tu mandasti (Communio)} \grecommentary[3pt]{Ps 118:4, 5} \gresetinitiallines{1} \grechangedim{maxbaroffsettextleft@nobar}{12 cm}{scalable} \grechangedim{spaceabovelines}{0.5cm}{scalable} \gresetlyriccentering{vowel}  \gregorioscore{graduale-chants/co--tu_mandasti--dominican--id_4828}  \vspace{5pt} \par{You have commanded Your commandments to be kept most diligently. O that my ways may be directed to keep Your justifications!} \newpage

\end{document}
