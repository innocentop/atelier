\noindent From the letter to the Hebrews \hfill 4:1-13

\begin{center}\textit{Let us strive to enter the Lord’s rest}\end{center}

\lettrine[lines=3,loversize=0.15]{W}{hile} the promise of entrance into his rest still holds, we ought to be fearful of disobeying lest any one of you be judged to have lost his chance of entering. We have indeed heard the good news, as they did. But the word which they heard did not profit them, for they did not receive it in faith.

It is we who have believed who enter into that rest, just as God said:

\vspace{5pt}
 “Then I swore in my anger,\par
    ‘They shall never enter into my rest.’”
\vspace{5pt}
Yet God’s work was finished when he created the world, for in reference to the seventh day Scripture somewhere says, “And God rested from all his work on the seventh day”; and again, in the place we have referred to, God says, “They shall never enter into my rest.”

Therefore, since it remains for some to enter, and those to whom it was first announced did not enter because of unbelief, God once more set a day, “today,” when long afterward he spoke through David the words we have quoted:

\vspace{5pt}
   “Today if you should hear his voice,\par
      harden not your hearts.”
\vspace{5pt}

Now if Joshua had led them into the place of rest, God would not have spoken afterward of another day. Therefore, a sabbath rest still remains for the people of God. And he who enters into God’s rest, rests from his own work as God did from his. Let us strive to enter into that rest, so that no one may fall, in imitation of the example of Israel’s unbelief.

Indeed, God’s word is living and effective, sharper than any two-edged sword. It penetrates and divides soul and spirit, joints and marrow; it judges the reflections and thoughts of the heart. Nothing is concealed from him; all lies bare and exposed to the eyes of him to whom we must render an account.

\vspace{5pt}

\textit{Or, according to} Proprium Officiorum Ordinis Prædicatorum, \textit{p. 702:}
