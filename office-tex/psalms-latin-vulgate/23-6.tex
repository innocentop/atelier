2. Quia ipse super mária fundáv\uuline{i}t \uline{e}um:~* et super flúmina præpar\uuline{á}vit \uline{e}um.\par 
3. Quis ascéndet in mont\uuline{e}m D\uline{ó}mini?~* aut quis stabit in loco s\uuline{a}ncto \uline{e}jus?\par 
4. Innocens mánibus et mund\uuline{o} c\uline{o}rde,~* qui non accépit in vano ánimam suam, nec jurávit in dolo próx\uuline{i}mo s\uline{u}o.\par 
5. Hic accípiet benedictiónem \uuline{a} D\uline{ó}mino:~* et misericórdiam a Deo, salut\uuline{á}ri s\uline{u}o.\par 
6. Hæc est generátio quærénti\uuline{u}m \uline{e}um,~* quæréntium fáciem D\uuline{e}i J\uline{a}cob.\par 
7. Attóllite portas, príncipes, \uline{ve}stras,~† et elevámini, portæ æt\uuline{e}rn\uline{á}les:~* et introíb\uuline{i}t Rex gl\uline{ó}riæ.\par 
8. Quis est iste R\uuline{e}x gl\uline{ó}riæ?~* Dóminus fortis et potens: Dóminus pot\uuline{e}ns in pr\uline{ǽ}lio.\par 
9. Attóllite portas, príncipes, \uline{ve}stras,~† et elevámini, portæ æt\uuline{e}rn\uline{á}les:~* et introíb\uuline{i}t Rex gl\uline{ó}riæ.\par 
10. Quis est iste R\uuline{e}x gl\uline{ó}riæ?~* Dóminus virtútum ipse \uuline{e}st Rex gl\uline{ó}riæ.\par 
11. Glória Patri, \uuline{e}t F\uline{í}lio,~* et Spirít\uuline{u}i S\uline{a}ncto.\par 
12. Sicut erat in princípio, et nunc, \uuline{e}t s\uline{e}mper,~* et in sǽcula sæcul\uuline{ó}rum. \uline{A}men.\par 
